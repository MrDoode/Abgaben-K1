\section{Theoretischer Hintergrund}
Die Inversionsreaktion von Rohrzucker (Saccharose) wird durch ein Proton katalysiert. 
Saccharose wird zu Glukose und Fruktose umgesetzt. 
\begin{equation}
\schemestart
\chemname{\chemfig{C_{12}H_{22}O_{11}}}{Saccharose} \arrow{->[$H^+$]}\chemname{\chemfig{C_6H_{12}O_6}}{Glukose} + \chemname{\chemfig{C_6H_{12}O_6}}{Fruktose}
\schemestop
\end{equation}
Die Mischung aus Fruktose und Glukose wird als  Invertzucker bezeichnet. 
Saccharose und der Invertzucker sind optisch aktiv. 
Die äquimolare Mischung von Fruktose und Glukose hat in Summe einen kleineren Drehwinkel als die Saccharose. 
So kann der Reaktionsfortschritt anhand der Verkleinerung des Drehwinkels der Lösung betrachtet werden. 
Saccharose dreht die Ebene nach rechts (positiv) und der Invertzucker nach links (negativ). 
Beim arbeiten mit einer verdünnten wässrigen Lösung kann das Wasser als konstant angesehen werden.
Die Reaktionsgeschwindigkeit $k$ lässt sich mit der Gleichung \ref{eq:ln-alpha} berechnen. 
Der linke Term wird in Abhängigkeit von der Zeit dargestellt.
Die Steigung einer Ausgleichsgerade gibt dann die Reaktionsgeschwindigkeitskonstante.
\begin{equation}
	\ln \left( \frac{\alpha_0 - \alpha_\infty}{\alpha - \alpha_\infty }\right) = k \cdot t 
\label{eq:ln-alpha}
\end{equation}
Der Fehler dieser Darstellung lässt sich über die Gleichung \ref{eq:dLNalpha} berechnen.
\begin{equation}
\Delta \ln \left( \frac{\alpha_0 - \alpha_\infty}{\alpha - \alpha_\infty }\right)  = \sqrt{\left(\frac{\Delta \alpha}{\alpha_0 - \alpha_\infty}\right)^2 +\left(\frac{\Delta \alpha \cdot \left(\alpha_0 -\alpha\right)}{\left(\alpha_\infty - \alpha\right)\cdot \left(\alpha_\infty-\alpha_0\right) }\right)^2 +\left(\frac{\Delta \alpha}{\alpha_\infty - \alpha}\right)^2}  \\
\label{eq:dLNalpha}
\end{equation}
Nach dem Arrhenius-Gesetz lässt sich aus der Änderung der Geschwindigkeitskonstante $k$, mit der Temperatur $T$ die Aktivierungsenergie berechnen, mit:
\begin{equation}
\ln (k) = -\frac{E_A}{R} \cdot \left( \frac{1}{T} \right) + \ln (A) \label{eq:EA}
\end{equation}
In der Darstellung werden dann die Fehlerkreuze benötigt.
Diese werden über die Gleichungen \ref{eq:dLNk} und \ref{eq:d1T} berechnet.
\begin{align}
\Delta \left(\ln (k)\right)&=\sqrt{\left(\frac{\delta \ln (k)}{\delta k}\cdot \Delta k \right)^2} = \frac{\Delta k}{k} \label{eq:dLNk}\\
\Delta \left(\frac{1}{T}\right)&=\sqrt{\left(\frac{\delta \frac{1}{T}}{\delta T}\cdot \Delta T \right)^2} = -\frac{\Delta T}{T^2}\label{eq:d1T}
\end{align}
Der Fehler für die Aktivierungsenergie und der Geschwindigkeitskonstante wird graphisch ermittelt über die Auswertung die Fehlerkreuze. 
Zur Bestimmung der Fehler werden jeweils zwei Extremgeraden eingezeichnet, welche jeweils durch $\frac{2}{3}$ der Fehlerboxen verläuft.
Außerdem wird $t_{1/2}$ mit der Gleichung \ref{eq:t0.5} ermittelt.
\begin{equation}
t_{1/2} = \frac{\ln (2)}{k} = \frac{\ln (2)}{A \cdot e^\frac{-E_A}{R\cdot T}}
\label{eq:t0.5}
\end{equation}
Die Fehler der Halbwertzeiten folgen aus der Fehlerfortpflanzung nach Gleichung \ref{eq:dt05}
\begin{align}
\Delta t_{1/2} &= \sqrt{\left(\frac{\delta t_{1/2}}{\delta A}\cdot \Delta A\right)^2 + \left(\frac{\delta t_{1/2}}{\delta E_A}\cdot \Delta E_A\right)^2 +\left(\frac{\delta t_{1/2}}{\delta T}\cdot \Delta T \right)^2}\\
\frac{\delta t_{1/2}}{\delta A} &= - \frac{1}{A^2} \cdot \frac{\ln (2)}{e^{- \frac{E_A}{R \cdot T}}}\\
\frac{\delta t_{1/2}}{\delta E_A} &= \frac{\ln (2)}{A\cdot R \cdot T\cdot e^{\left(- \frac{E_A}{R \cdot T}\right)}} \\
\frac{\delta t_{1/2}}{\delta T} &= - \frac{E_A}{R\cdot T^2}\cdot \frac{\ln (2)}{A\cdot e^{\left(- \frac{E_A}{R \cdot T}\right)}}\\
\Rightarrow \Delta t_{1/2} &= \sqrt{\left(\frac{\ln (2)}{A\cdot e^{- \left(\frac{E_A}{R\cdot T}\right)}}\right)^2 \cdot \left(\left(\frac{\Delta A}{A}\right)^2 + \left(\frac{\Delta E_A}{R\cdot  T}\right)^2 + \left(\frac{E_A \cdot \Delta T}{R\cdot T^2}\right)^2\right)}
\end{align}
