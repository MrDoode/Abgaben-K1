\section{Durchführung}
Der Thermostat war schon vor dem Versuch eingeschaltet und auf \qty{25}{\degreeCelsius} eingestellt. 
Danach wurde der Nullpunkt des Polarimeters mit destilliertem Wasser kontrolliert. 
So soll der Größtfehler des Polarimeters abgeschätzt werden.
Anschließend wurden etwa \qty{100}{\milli\liter} einer Rohrzuckerlösung und etwa \qty{100}{\milli\liter} einer HCL-Lösung abgefüllt und unabhängig voneinander im Wasserbad auf \qty{25}{\degreeCelsius} temperiert. 
Dann wurden jeweils \qty{25}{\milli\liter} der HCL-Lösung und der Rohrzuckerlösung in einem Becherglas vermischt. 
Gleichzeitig wurde die Zeitmessung gestartet. 
Danach wurde rasch mit einer Pipette genug Flüssigkeit von der Mischung entnommen, um das Polarimeterrohr blasenfrei zu befüllen. 
Eine Messung dauerte \qty{20}{\minute}, dabei wurden in den ersten \qty{10}{\minute} alle \qty{30}{\second} abgelesen und danach alle \qty{60}{\second}. 
Die Messungen wurden für \qty{30}{\degreeCelsius} und \qty{35}{\degreeCelsius} wiederholt. 
Für die Messung von $\alpha_\infty$ wurde zuerst \qty{25}{\milli\liter} der Messlösung  im Wasserbad auf \qty{70}{\degreeCelsius} erhitzt und anschließend auf \qty{20}{\degreeCelsius} abgekühlt. 
Danach wurde das Polarimeter blasenfrei mit der Messlösung gefüllt und der Wert für $\alpha_\infty$ abgelesen.
